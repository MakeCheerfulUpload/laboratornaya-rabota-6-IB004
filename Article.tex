\newpage
\vspace*{-1.6cm}
\begin{flushleft}
\begin{flushleft}
22
\end{flushleft}
\vspace*{-1.2cm}
{\center{\textbf{К В А Н T $\cdot$}  2 0 1 5 / № 3}}

\begin{multicols}{2}
\parindent=0,0cm % смена абзацного отступа
\abovedisplayskip = 0.1cm%отступ перед формулой
\belowdisplayskip = -0.1cm%отступ после формулы


\textit{с длиной волны $\lambda$ определяется групповой скоростью $u = d\omega/dk$. 
Групповая скорость u может быть найдена по формуле Эйлера: $\displaystyle  u = v - \lambda \frac{dv}{d\lambda}.$}

Учитывая, что $v = \omega / k$, из закона дисперсии находим
зависимость фазовой скорости от частоты: 
$$v = \frac{g}{\omega}.$$

Из формулы Эйлера для групповой скорости получаем
$$u = v - \lambda \frac{dv}{d\lambda} = \frac{1}{2}v = \frac{g}{2\omega}.$$

Если расстояние до места падения метеорита $L$, а
регистрация волн началась через время $\tau$ после падения
метеорита, то время прихода групп волн с частотой
$\omega = 2\pi/T$ равно $t' = t + \tau$, т.е.
$$\frac{L}{u} = \frac{g}{g/(2\omega)} = t + \tau \text{, или } \omega = \frac{g(t+\tau)}{2L}.$$

Получается, что частота $\omega$ линейно растет со временем, причем угловой коэффициент прямой $\omega(t)$ равен $A = g/(2L)$. Построим график зависимости $\omega = \omega(t)$,
соответствующий таблице 2.
\vspace*{-0.3cm}

{\raggedleft \textit{Таблица 2}\\}
\begin{adjustbox}{width = 0.48\textwidth}
\begin{tabular}{|c|c|c|c|c|c|c|c|c|c|c|}
\hline
$t,$ ч & 0 & 3 & 6 & 9 & 12 & 15 & 18 & 21 & 24 & 27\\
\hline
$\omega$, c$^{-1}$&1,10& 1,26 & 1,46 & 1,70 & 1,90 & 2,02 & 2,24 & 2,4 & 2,5 & 2,73\\
\hline
\end{tabular}
\end{adjustbox}

\includegraphics[scale = 0.4]{pic1.png}

График, приведенный на рисунке, хорошо описывается
прямой $\omega(t) = At + B$ угловым коэффициентом
$$A = \frac{\Delta\omega}{\Delta t} = 0,06 \text{ с$\sfrac{-1}{\text{ч}}$}.$$
\vspace*{0.07cm}

Отсюда находим расстояние до места падения спут-
ника на землю:
$$L = \frac{g}{2A} \approx 300 \text{км}.$$

Метеорит упал за $\tau = B/A = $18,5 ч до начала наблюдений. 
Учитывая, что наблюдения за волнением начались
в 12:00, момент падения метеорита соответствует вре-
мени 17:30 предшествующих дню наблюдения суток.

{\raggedleft \textit{А.Гуденко}\\}
\end{multicols}

\begin{figure}[h]
\centering
\includegraphics[width=15cm,height=1.4cm]{pic2.png}
\end{figure}
\vspace*{-1.4cm}
\begin{center}
\textbf{\large Н\:А\:М\; \;П\:И\:Ш\:У\:Т}
\vspace*{-0.0cm}
\end{center}

\begin{multicols}{2}
\begin{center}
\setmainfont{Arial}
\textbf{\large Глиняные гири}
\vspace*{-0.0cm}
\end{center}
\parindent=0,2cm
Не секрет, что математика – вовсе не сухая и скучная
наука. В ней много интересных задач, и бывает, что
впечатление от решения красивой задачи запоминается на
всю жизнь.

О таком ярком моменте из своих школьных лет написал
нам наш читатель из города Пересвет Московской области
Данил Владимирович Поташников, ветеран Великой оте-
чественной войны. Вот несколько его строк о себе:

«В 1961 году закончил МАИ очно. В 1999 году заочно
освоил пятигодичный курс Открытого университета Из-
раиля. Не пропустил ни одну лекцию из цикла «Академия
телеканала «Культура».

А вот выдержка из его письма о запомнившейся задаче:

«Когда я учился в пятом классе (а это было в городе
Каменка Черкасской области на Украине в 1936 году),
учитель математики записал на доске домашнее задание и
попросил дополнительно решить головоломку.

\textit{На Украине в XIX веке гири для рычажных весов
изготавливались и самодельные – из глины. Самая
большая была пудовая (40 фунтов). По дороге на
ярмарку пудовая гиря упала с воза и разбилась на четыре
части. Оказалось, что этими частями можно взвесить
на рычажных весах любые покупки весом от одного до
сорока фунтов. Суть задания: найти вес каждой части.}

Никогда не забуду ту бессонную ночь!

Когда я назвал вес каждой части: 1, 3, 9, 27, учитель
попросил выйти к доске и пояснить ответ.

Один фунт – нелогично использовать две части для
определения одного фунта.

Три фунта – «1» и «3» позволят взвесить 1, 2, 3 и 4
фунта.

Девять фунтов – сможем взвесить от 5 до 13 фунтов.

Двадцать семь фунтов – сможем взвесить от 14 до 40
фунтов.

На одной из последних встреч с учениками 6-го класса
я попросил решить эту головоломку. Я сообщил детям
свой телефон и обещал подарок тому, кто первый найдет
решение.

Увы!»

Предлагаем нашим читателям справиться с таким обоб-
щением этой головоломки, ставшим классической олим-
пиадной задачей:

\textit{Докажите, что с помощью n гирь массами $1, 3, 9,
\ldots,3^{n-1}$ кг можно взвесить на чашечных весах любой
предмет массой $\displaystyle M \le \frac{3^n-1}{2}$ кг, (М – целое число, гири
можно класть на обе чаши весов).}

В завершение приведем еще одну цитату из письма
Д.В.Поташникова:

«В этом году по просьбе детей и внуков я написал свои
воспоминания, которые закончил словами «Я живу, пока
познаю».

\end{multicols}
\end{flushleft}
